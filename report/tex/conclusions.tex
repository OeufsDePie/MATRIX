\section{Realized parts}
We will present in this section what is currently usable in the final version of the application.
\subsection*{Workspace Management}
When the user launch the application, he will be asked to create a new workspace. You cannot use the application without first creating a workspace, like in eclipse. 
When a workspace is created, in a specific folder, you can create another workspace, delete your current workspace or change to another workspace. When you are done with
a workspace, you can save it and re-open it later.

In a workspace, you can create a scene or multiple scenes. A scene corresponds to a set of photos, like a real scene. You can change your current scene to another one already created or delete your current scene. Note
that when you create a workspace, a default scene is created so you don't have to create a scene.

\subsection*{Pictures Handling}
When you are in your workspace, you can import pictures, either from a specific folder or from a camera. If a camera is connected, the application will detect it and display its name. If you choose to import photos from
your camera, the application will import \textbf{all} your photos as thumbnails. 

You can then manage your pictures, select the one you want to import as real photos and not as thumbnails by setting their state as \textit{Discarded} or \textit{New}, if you imported them from your camera. The is on 
the left of the screen a list with your pictures and a preview above the list. You can select pictures, drag and drop them as you want to sort them. You can filter them by their state.

\subsection*{Map View}
If you switch on the \textit{Show the map} button, a map centered on your pictures will appear (if you have pictures with GPS exif data, otherwise the map be centered on the point of coordinates 0.0,0.0). You can 
navigate on the map as you want, or click on a picture in the preview list to center the map on it. You can also click on a point on the map to select the corresponding picture in the list. Note that each picture state has
its own color code, see the documentation for further details. There is a good interaction between the preview list and the map. 

\subsection*{Reconstruction}
When you are all set for the reconstruction, meaning that you have selected a set of pictures you want to use for the recontruction, you can launch the 3D reconstruction. In the current version of the application,
this will freeze the application and you will have to wait for the reconstruction to be done to be able to re-use the application. Once the reconstruciton is done, you will see the 3D rendering of your scene wetween the map
and the preview list.

\subsection*{3D Renderer}
The 3D renderer currently only display a point cloud when the application gives him one. Note that the point of view from where you see the point cloud is the exact location of one of your camera. There 
are still not any interactions between the 3D renderer and the rest of the application, see the documentation for further details about that.

\section{Missing features}
While the application was supposed to be a prototype and was not to be a hundred percent usable and stable, there are still essentials features that are missing. Here a list of those features, classified by their importance :
\begin{description}
\item[Dynamic behavior between Pictures selection, Map view and 3D View] There are many interactions between the preview list and the map viewer like said previously. One important feature that evry user would naturally
expect is the same type of interaction between those two modules and the 3D view. Clicking on a picture would not only change the current view of the map, but also the current view on the 3D view.
\item[3D Navigation] Being able to navigate inside the 3D world to see the point cloud is also something any user would expect to be present in the application. 
\item[Automated Reconstruction] As said before, you have to manually launch the reconstruction when you have selected your pictures. What the client expected from us was that the application would automatically launch
a new reconstruction when a new picture, in the right state, has been added. 
\item[Reconstruction Feedback] When the user launches the reconstruction, he currently never knows if there are problems with the reconstruction, or its progress state. A feedback, simply like logs, is also something missing.
\end{description}

\section{Insights}
Here we will try to conclude and describe a bit our feelings about this project. The first question to be asked at the end of this project is : \textbf{Did we reach client expectations ?} . 
As said before, we implemented most of the functionalities we were asked to do, but some import features are still missing. But this project being our first big group project, we have encountered many isssues we did not expect
to encounter, and those issues made us late on what was initially schedueled. But we have learned from those issues and we are now stronger and more prepared for future projects we might have, and we think this is the 
main objective of the \textit{projet long}. And given the fact that none of use knew any of the languages used for the project and that the client knew about that, we can say that we have reached client expectations.

But, while reading this, one has to keep in mind that the main goal of this project was to build a prototype that our client could easily re-use for demonstration or for snippets of code. We have also build a solid 
architecture around the project, a good documentation which explain communications between modules of this architecture	and so we have proved that the application could be done. The application can now be re-used that way and is 
also extendable.


