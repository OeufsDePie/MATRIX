\section{Objectives}
The core of this project was to develop a Graphical User Interface
gathering a set of existing tools in a simple work flow. It aims at
automating and easing a common \textsc{3d} reconstruction process,
used in the cinema industry. The process consists of taking hundred of
photos of a large object and feed them to a \textsc{3d} reconstruction
library to get a \textsc{3d} model of the object. In the context of
the European project \emph{popart} under which is developed the
\textsc{3d} reconstruction library \emph{openMVG}, we are trying to
set up a prototype that may be used as a proof-of-concept for the
final user work flow. 

The major improvement on the application user pipeline over the
existing work flow is that the \textsc{3d} reconstruction can actually
be made on-set, while the operator is taking pictures of the
scene. Doing so, we avoid losing time by traveling back-and-forth
between the scene and the studio, and we allow for a quick feedback on
the pictures being taken.

Ensuring a global pipeline, we have also implemented as many features
as possible to enhance the completeness of the application. The whole
application has been developed using \code{Python}, \code{C++} and
\code{Qml} and various libraries and framework described in a related
document.

On the other hand, the project was a real opportunity to face and cope
with management challenges. Well above the technical and practical
experience we got, we learned a lot about task managing, continuous
risks analysis and scheduling. The project ought to give us an
overview of future situations we may encounter.

\section{Supplies}
The core of the application relies on a specific reconstruction
library called \emph{openMVG}. It is a cornerstone of the project; it
has been provided by the client via a \emph{GitHub} repository. Given
that, we were offered various indications, whether it was about which
language to use or about libraries that could prove useful.

\section{Deliverables}
We delivered our prototype version as \emph{GitHub} repository. In
that way, client is able to clone the repository and launch the
application from a defined milestone. Plus, he might be able to carry
on our work by taking the lead onto this repository.  Also, the
repository allow us to gather source code on one side, and a
documentation on the other side. The documentation is by the by
compiled into \code{html} documents and it is therefore easily
accessible online.

