\section{Objectives}
During this project, we have been leading to develop a Graphical User Interface 
which holds in background a complex processus pipeline. It aims at making automatic
and easier a common reconstruction process, used in the cinema industry in order
to compute \textsc{3d} models from a set of \textsc{2d} pictures. In the context
of the european project \emph{Popart},  we are trying to set up a small prototype
that may be shown as a proof of concept during further presentations of the project.

\noindent
Ensuring a global pipeline, we have also implemented as many as possible features
to enhance the completeness of the application. The whole application has been
developed using \code{python}, \code{C++} and \code{Qml} and various libraries 
and framework described in a related document. 

\noindent 
On the other hand, the project was a real opportunity to face and cope with 
management challenges. Well above the technical and practical experience we've get, 
we have learnt a lot about task managing, continuous risks analysis and scheduling.
The project ought to give us an overview of future situations we may encounter.  

\section{Supplies}
The core of the application relies on a specific reconstruction library called 
\emph{openMVG}. It is indeed an essential cornerstone of the project; it has 
been supplied by the client via a \emph{GitHub} repository. Given that, various 
indications were offered either about languages to use or useful libraries.

\section{Deliverables}
As a product, we delivered our prototype version in the form of a \emph{GitHub} 
repository. In that way, client is able to clone the repository and launch the 
application from a defined milestone. Plus, he might be able to carry on our 
work by taking the lead onto this repository. 
Also, the repository allow us to gather source code on one side, and a documentation
on the other side. The documentation is by the by compiled into \code{html} 
documents and it is therefore easily accessible online. 

